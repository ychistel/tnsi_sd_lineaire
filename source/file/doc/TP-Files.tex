\documentclass[11pt,a4paper]{article}

\usepackage{style2017}
\usepackage{hyperref}

\hypersetup{
    colorlinks =false,
    linkcolor=blue,
   linkbordercolor = 1 0 0
}
\newcounter{numexo}
\setcellgapes{1pt}

\begin{document}



\begin{NSI}
{TP}{Jouer à la bataille}
\end{NSI}


\section{Présentation du jeu}

Le jeu de cartes de la bataille oppose 2 joueurs. Chaque joueur possède un tas de cartes qu'il ne voit pas. Il n'a pas le droit de les mélanger et ne peut accéder qu'à la carte située au-dessus de son tas. À chaque tour, les joueurs retournent la carte au-dessus du tas et le joueur qui a la carte la plus forte remporte la manche. Les deux cartes ramassées sont placées en-dessous du tas de cartes du joueur vainqueur. \medskip

En cas d'égalité, il y a bataille. Les 2 joueurs doivent retirer chacun 1 carte de leur tas et la poser sur la carte au tapis et retirer à nouveau une carte pour remporter la manche. Les batailles peuvent s'enchainer !

Le premier joueur qui n'a plus de cartes a perdu la bataille.

\section{Le paquet de cartes}

Une carte est représenté par un tuple (valeur, couleur).

\begin{itemize}
\item Les valeurs des cartes sont le 1 pour l'as, le 2 pour le deux, etc ; le valet est représenté par le nombre 11, la dame par le 12 et le roi par le 13.
\item Les couleurs sont les chaines de caractères "coeur", "carreau", "trèfle" et "pique".
\end{itemize}

Un paquet de cartes est une liste contenant 32 ou 52 cartes.


\begin{enumerate}

\item Écrire la fonction \textbf{creer\_paquet(n=32)} qui prend en paramètre le nombre \textbf{n} de cartes et renvoie la liste contenant toutes les cartes d'un jeu. Le paquet de cartes sera affecté à la variable \textsf{paquet}.

\item La méthode \textsf{shuffle}(liste) du module \textsf{random} permet de mélanger aléatoirement les valeurs de la liste. Attention, c'est bien la liste qui est mélangée, pas une copie !

Créer une fonction melanger(paquet) pour mélanger les cartes.

\item Les tas de cartes de chaque joueur sont de type \textsf{file}. Créer une fonction \textbf{distribuer(paquet)} qui va enfiler les cartes du paquet dans 2 files tasA et tasB alternativement. Les deux tas seront renvoyés par la fonction.
\end{enumerate}

\section{Le jeu de la bataille}
Le jeu se déroule en trois phases :
\begin{enumerate}
\item On tire (défile) une carte \textbf{a} du tasA et une carte \textbf{b} du tasB;
\item On compare les 2 cartes:
\begin{itemize}
\item Si la carte \textbf{a} est plus forte que la carte \textbf{b}, alors on met les 2 cartes sous le tasA;
\item sinon, si la carte \textbf{b} est plus forte que la carte \textbf{a}, alors on met les 2 cartes sous le tasB;
\item sinon (les cartes sont aussi fortes) il y a bataille.
\end{itemize}
\item En cas de bataille, on enfile les cartes du tapis et une autre carte de chaque paquet dans une fille temporaire (tapis). Ensuite on recommence avec 2 nouvelles cartes et la plus forte remporte le tout.
\end{enumerate}

\newpage
On donne l'algorithme du jeu :

\begin{center}
\begin{tabular}{|p{14cm}|}
\hline
On crée une file vide \textbf{tapis} (cartes en cas de bataille)\\
Tant que les tas A et B ne sont pas vides:\\
\hspace{1cm}on défile le tasA (a) \\
\hspace{1cm}on défile le tasB (b) \\
\hspace{1cm}si $a>b$ :\\
\hspace{2cm}on enfile a et b dans le tasA\\
\hspace{2cm}on récupère tout le tapis dans le tasA (suite à bataille)\\
\hspace{1cm}sinon si $a<b$:\\
\hspace{2cm}on enfile a et b dans le tasB\\
\hspace{2cm}on récupère le tapis dans le tasB (suite à bataille)\\
\hspace{1cm}sinon: (bataille)\\
\hspace{2cm}on enfile a et b dans tapis\\
\hspace{2cm}on défile tasA et on enfile tapis\\
\hspace{2cm}on défile tasB et on enfile tapis\\
Fin de la boucle tant que (un des tas est vide)\\
si paquet A vide:\\
\hspace{1cm}afficher "A a perdu, B gagne"\\
sinon:\\
\hspace{1cm}afficher "B a perdu, A gagne"\\
\hline
\end{tabular}
\end{center}

Écrire une fonction \textbf{jouer()} programmant l'algorithme ci-dessus.

\section{Programme principal}
Voici l'algorithme du programme principal:
\begin{center}
\begin{tabular}{|p{14cm}|}
\hline
Créer une file vide \textbf{tasA}\\
Créer une file vide \textbf{tasB}\\
Créer un paquet de cartes \textbf{paquet}\\
Mélanger le paquet\\
Distribuer les cartes aux joueurs A et B\\
Lancer le jeu avec la fonction jouer().\\
\hline
\end{tabular}
\end{center}

\begin{enumerate}
\item Écrire en python le programme principal et l'exécuter.
\item Le programme n'affiche rien. Ajouter des affichages pour:
\begin{itemize}
\item voir les cartes posées par les joueurs A et B;
\item voir qui remporte la manche;
\item voir lorsqu'il y a bataille et les cartes dans le tas;
\item voir le nombre de cartes de chaque joueur après chaque manche?
\end{itemize}

\end{enumerate}

\end{document}
